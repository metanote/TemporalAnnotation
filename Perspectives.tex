\chapter{Conclusion et perspectives}
\paragraph{}
Comme toute étude de recherche, les portes sont toujours ouvertes pour des améliorations et des adaptations. Notre logiciel est disponible sur le gestionnaire de version GitHub en version open source sur mon compte\footnote{https://github.com/metanote/Extraction} ce qui permet d'ouvrir les pistes à d'autres personnes pour l'utiliser et l'améliorer. 
\section*{Améliorations possibles}
\paragraph{}
Ce logiciel est un outil d'extraction de propriétés à partir de DBpedia, permet d'annoter temporellement des triplets DBpedia. Actuellement, cet outil dépend de l'avis d'un expert pour valider les résultats de notre algorithme. Il serait intéressant de rendre toute cette procédure de modélisation automatique. Aussi, on peut intégrer une procédure d'apprentissage, en cherchant à classifier les propriétés DBpedia sous forme de deux classes par exemple (PropWithResult, PropWithoutResult) avec un algorithme d'apprentissage automatique comme (SVM ou Adaboost) pour la procédure de validation. Aussi, nous pouvons chercher seulement les triplets que l'on veut annoter dans DBpedia, puis à partir des faits de ces triplets, nous cherchons les annotations temporelles dans les sauvegardes de Wikipédia et Wikidata.
\subparagraph{}
Dans cette étude, nous avons travaillé avec deux préfixes temporels (Date, Year). Nous pouvons chercher d'autres préfixe qui peuvent nous aider à repérer plus de propriétés temporelles dans DBpedia. Nous avons tourné notre algorithme sur $1992$ propriétés DBpedia que nous avons extraites. Il serait intéressant de tourner cet algorithme sur une autre base de faits qui contient plus de propriétés DBpedia.
Nous avons développé un prototype dans lequel nous avons cherché seulement les propriétés qui dépendent d'un point de temps particulier.
Dans notre hypothèse de base, nous voulons aussi chercher les propriétés qui sont vraies dans un intervalle de temps, des propriétés temporelles comme ($préfixeStartDate$, $préfixeEndDate$), mais dans le fichier de propriétés DBpedia nous avons repéré seulement $7$ couples de propriétés qui vérifie cette condition. Cela donne seulement $7$ intervalles de temps possibles sur $1992$.
\subparagraph{}
La procédure de la fouille est intéressante pour une quantité de données de masse. Nous avons formé avec les suffixes temporels (Year, Date) $305$ couples de propriétés à partir de $1992$ propriétés. Sur les $20$ premiers couples de propriétés, $4$ couples donnent des quadruplets valides et à l'aide du choix d'un expert, nous avons récupéré $26$ fichiers de triplets annotés. 
\newpage
\section*{Conclusion}
\paragraph{}
Le stage réalisé a été enrichissant de plusieurs façons. En effet, bien que ce stage soit un stage orienté recherche, il comporte de nombreux aspects d'ingénierie. Les recherches effectuées sont des réponses aux besoins qui ont été accordées dans d'autres travaux de recherche sur l'annotation temporelle des triplets RDF et qui est également lié aux travaux de mes tuteurs de stage. Les réponses sont également dirigées vers du concret notamment l'élaboration d'une application Java. 
\subparagraph{}
Cette dualité recherche-ingénierie a été motivante et m'a permis d'affiner une certaine ouverture d'esprit et d'améliorer mes connaissances dans le domaine du Web sémantique.
En plus, j'ai assisté à des présentations dans les domaines du Web sémantique et Big Data. Cela m'a permis de voir les études et les travaux dans ces domaines. La rencontre de différentes visions est toujours intéressante et m'a permis dans ce cas de situer l'informatique et l'utilisation. Par ailleurs, le travail dans un milieu de recherche m'a permis de chercher des réponses à des questions et résoudre des problématiques intéressantes.
\subparagraph{}
Durant cette étude, j'ai appris que la bonne modélisation des connaissances permet de rendre les informations plus utiles et elle peut éviter beaucoup de redondances. Actuellement on parle plus de l'énorme quantité de données ou les données massives $Big$ $Data$ sur le Web. Derrière ces mots se cachent l’incroyable quantité de données disponible notamment sur internet, et surtout la manière dont on peut les traiter pour obtenir des informations utiles. On se rend compte de la mauvaise gestion, la duplication, la perte de l'information et la difficulté liée à la recherche de ces informations. C'est pour cela, nous cherchons toujours à optimiser l'usage de ces métadonnées et structuré les données sur le Web.
\subparagraph{}
Enfin, l'étude d'annotation temporelle des données DBpedia m'a permis d'approfondir mes connaissances dans le domaine du Web sémantique et de prendre conscience de son importance. Je suis dorénavant intimement convaincu que les recherches dans le Web sémantique vont permettre d'évoluer le Web et de concrétiser la vision de Tim Berners-Lee.   
